%2multibyte Version: 5.50.0.2960 CodePage: 1252

\documentclass{article}
%%%%%%%%%%%%%%%%%%%%%%%%%%%%%%%%%%%%%%%%%%%%%%%%%%%%%%%%%%%%%%%%%%%%%%%%%%%%%%%%%%%%%%%%%%%%%%%%%%%%%%%%%%%%%%%%%%%%%%%%%%%%%%%%%%%%%%%%%%%%%%%%%%%%%%%%%%%%%%%%%%%%%%%%%%%%%%%%%%%%%%%%%%%%%%%%%%%%%%%%%%%%%%%%%%%%%%%%%%%%%%%%%%%%%%%%%%%%%%%%%%%%%%%%%%%%
%TCIDATA{OutputFilter=LATEX.DLL}
%TCIDATA{Version=5.50.0.2960}
%TCIDATA{Codepage=1252}
%TCIDATA{<META NAME="SaveForMode" CONTENT="1">}
%TCIDATA{BibliographyScheme=Manual}
%TCIDATA{Created=Sunday, October 11, 2015 14:04:36}
%TCIDATA{LastRevised=Friday, October 23, 2015 22:22:47}
%TCIDATA{<META NAME="GraphicsSave" CONTENT="32">}
%TCIDATA{<META NAME="DocumentShell" CONTENT="Standard LaTeX\Blank - Standard LaTeX Article">}
%TCIDATA{CSTFile=40 LaTeX article.cst}

\newtheorem{theorem}{Theorem}
\newtheorem{acknowledgement}[theorem]{Acknowledgement}
\newtheorem{algorithm}[theorem]{Algorithm}
\newtheorem{axiom}[theorem]{Axiom}
\newtheorem{case}[theorem]{Case}
\newtheorem{claim}[theorem]{Claim}
\newtheorem{conclusion}[theorem]{Conclusion}
\newtheorem{condition}[theorem]{Condition}
\newtheorem{conjecture}[theorem]{Conjecture}
\newtheorem{corollary}[theorem]{Corollary}
\newtheorem{criterion}[theorem]{Criterion}
\newtheorem{definition}[theorem]{Definition}
\newtheorem{example}[theorem]{Example}
\newtheorem{exercise}[theorem]{Exercise}
\newtheorem{lemma}[theorem]{Lemma}
\newtheorem{notation}[theorem]{Notation}
\newtheorem{problem}[theorem]{Problem}
\newtheorem{proposition}[theorem]{Proposition}
\newtheorem{remark}[theorem]{Remark}
\newtheorem{solution}[theorem]{Solution}
\newtheorem{summary}[theorem]{Summary}
\newenvironment{proof}[1][Proof]{\noindent\textbf{#1.} }{\ \rule{0.5em}{0.5em}}
\input{tcilatex}
\begin{document}


\section{New Keynesian Model with Sticky prices}

\subsection{Households}

Utility function is

\begin{equation}
E_{0}\sum_{t=0}^{\infty }\beta ^{t}\left[ \ln c_{t}-\frac{L_{t}^{\eta }}{%
\eta }\right] .  \label{Utility function}
\end{equation}%
where $c_{t}$ is consumption, $L_{t}$ is labour supply, $\beta $ is discount
factor.

Budget constraint in real term are

\begin{eqnarray}
c_{t}+d_{t} &=&w_{t}L_{t}+\frac{R_{t}d_{t-1}}{\pi _{t}}+T_{t}+\Pi _{t}
\label{Intertemporal constraint} \\
T &=&c+k-wL-Rk-\Pi
\end{eqnarray}%
where $R_{t}$ is nominal interest rate from $t-1$ to $t$, $d_{t}$ is
households lending, $\pi _{t}=\frac{P_{t}}{P_{t-1}}$ is inflation, $w_{t}$
is real wage, $\Pi _{t}$ is lump-sum profit received from retailers, $T_{t}$
is public transfers. The profit terms can be expressed as%
\[
\Pi _{t}=\left( 1-\frac{1}{X_{t}}\right) Y_{t}. 
\]

Households' problem is to maximize lifetime utility s.t. the budget
constraint,

\[
\mathcal{L=}E_{0}\sum_{t=0}^{\infty }\beta ^{t}\left[ \ln c_{t}-\frac{%
L_{t}^{\eta }}{\eta }+\lambda _{t}\left( w_{t}L_{t}+\Pi _{t}+\frac{%
R_{t}d_{t-1}}{\pi _{t}}+T_{t}-c_{t}-d_{t}\right) \right] . 
\]

FOCs are%
\[
\frac{1}{c_{t}}=\lambda _{t}, 
\]%
\[
-L_{t}^{\eta -1}+\lambda _{t}w_{t}=0, 
\]%
\[
-\lambda _{t}+\beta E_{t}\lambda _{t+1}\frac{R_{t+1}}{\pi _{t+1}}=0. 
\]

Solution is

\begin{equation}
\frac{1}{c_{t}}=\beta E_{t}\left[ \frac{R_{t+1}}{\pi _{t+1}c_{t+1}}\right] .
\label{Euler equation}
\end{equation}%
\begin{equation}
c_{t}L_{t}^{\eta -1}=w_{t}.  \label{Labour supply}
\end{equation}

\subsection{Entrepreneurs}

Competitive entrepreneurs, borrow capital, hire labour from households,
produce intermediate output%
\[
Y_{t}=A_{t}k_{t}^{\alpha }L_{t}^{1-\alpha }. 
\]

They%
\[
\max \frac{P_{t}^{w}}{P_{t}}A_{t}k_{t}^{\alpha }L_{t}^{1-\alpha
}-r_{k,t}k_{t}-w_{t}L_{t}, 
\]%
where $P_{t}^{w}$ is the price selling to retailers, and $r_{k,t}$ is
marginal product of capital. We denote $X_{t}=\frac{P_{t}}{P_{t}^{w}}$ as
price mark up.

FOCs are%
\begin{equation}
\alpha \frac{A_{t}k_{t}^{\alpha -1}L_{t}^{1-\alpha }}{X_{t}}=r_{k,t},
\label{MPK}
\end{equation}

\begin{equation}
\frac{\left( 1-\alpha \right) A_{t}k_{t}^{\alpha }L_{t}^{\alpha }}{X_{t}}%
=w_{t}  \label{MPL}
\end{equation}

\subsection{Capital accumulation}

Capital accumulation with adjustment cost%
\[
k_{t+1}=\left( 1-\delta \right) k_{t}+i_{t}. 
\]

Resource constraint%
\[
Y_{t}=c_{t}+i_{t}. 
\]

\subsection{Retailers}

\[
\beta \widehat{\pi }_{t+1}=\widehat{\pi }_{t}+\frac{\left( 1-\theta \beta
\right) \left( 1-\theta \right) }{\theta }\widehat{X}_{t}. 
\]%
where $X=\frac{\varepsilon }{\varepsilon -1}$ at the steady state.

\subsection{System of equations}

\[
\frac{1}{c_{t}}=\beta E_{t}\left[ \frac{R_{t+1}}{\pi _{t+1}c_{t+1}}\right] 
\]

\[
c_{t}L_{t}^{\eta -1}=w_{t} 
\]

\[
\alpha \frac{A_{t}k_{t}^{\alpha -1}L_{t}^{1-\alpha }}{X_{t}}=r_{k,t} 
\]

\[
\frac{R_{t}}{\pi _{t}}=r_{k,t}+1-\delta 
\]

\[
\frac{\left( 1-\alpha \right) A_{t}k_{t}^{\alpha }L_{t}^{\alpha }}{X_{t}}%
=w_{t} 
\]

\[
Y_{t}=A_{t}k_{t}^{\alpha }L_{t}^{1-\alpha } 
\]

\[
k_{t+1}=\left( 1-\delta \right) k_{t}+i_{t} 
\]

\[
Y_{t}=c_{t}+i_{t} 
\]

\section{Learning part}

\subsection{Consumption function}

We start from intertemporal constraint (\ref{Intertemporal constraint})%
\[
c_{t}+d_{t}=w_{t}L_{t}+\frac{R_{t}d_{t-1}}{\pi _{t}}+T_{t}+\Pi _{t}. 
\]

It can be written as%
\begin{equation}
\Lambda _{t}+\frac{R_{t}d_{t-1}}{\pi _{t}}=d_{t},  \label{1}
\end{equation}

where%
\[
\Lambda _{t}=w_{t}L_{t}+T_{t}+\Pi _{t}-c_{t}. 
\]

Rewite (\ref{1}) one step ahead

\begin{equation}
\Lambda _{t+1}+\frac{R_{t+1}d_{t}}{\pi _{t+1}}=d_{t+1}  \label{2}
\end{equation}

Substitute into (\ref{1})

\begin{eqnarray}
\frac{R_{t+1}}{\pi _{t+1}}\Lambda _{t}+\frac{R_{t+1}}{\pi _{t+1}}\frac{R_{t}%
}{\pi _{t}}d_{t-1}+\Lambda _{t+1}+\frac{R_{t+1}d_{t}}{\pi _{t+1}} &=&\frac{%
R_{t+1}}{\pi _{t+1}}d_{t}+d_{t+1}  \nonumber \\
\frac{R_{t+1}}{\pi _{t+1}}\Lambda _{t}+\Lambda _{t+1}+\frac{R_{t+1}}{\pi
_{t+1}}\frac{R_{t}}{\pi _{t}}d_{t-1} &=&d_{t+1}  \label{3}
\end{eqnarray}

Rewrite (\ref{1}) two steps ahead

\begin{equation}
\Lambda _{t+2}+\frac{R_{t+2}}{\pi _{t+2}}d_{t+1}=d_{t+2}  \label{4}
\end{equation}

Substitute into (\ref{3})

\begin{equation}
\frac{R_{t+2}}{\pi _{t+2}}\frac{R_{t+1}}{\pi _{t+1}}\Lambda _{t}+\frac{%
R_{t+2}}{\pi _{t+2}}\Lambda _{t+1}+\Lambda _{t+2}+\frac{R_{t+2}}{\pi _{t+2}}%
\frac{R_{t+1}}{\pi _{t+1}}\frac{R_{t}}{\pi _{t}}d_{t-1}=d_{t+2}  \label{5}
\end{equation}

So on so forth, we have%
\[
\Lambda _{t}+\frac{\pi _{t+1}}{R_{t+1}}\Lambda _{t+1}+...+\left(
\dprod\limits_{i=1}^{n}\frac{\pi _{t+i}}{R_{t+i}}\right) \Lambda _{t+j}+%
\frac{R_{t}}{\pi _{t}}d_{t-1}=\left( \sum_{i=1}^{j}\frac{\pi _{t+i}}{R_{t+i}}%
\right) d_{t+j}. 
\]

\begin{equation}
\Lambda _{t}+\sum_{j=1}^{\infty }\left( \dprod\limits_{i=1}^{j}\frac{\pi
_{t+i}}{R_{t+i}}\right) \Lambda _{t+j}+\frac{R_{t}}{\pi _{t}}d_{t-1}=0,
\label{6}
\end{equation}%
where%
\begin{equation}
\Lambda _{t+j}=w_{t+j}L_{t+j}+T_{t+j}+\Pi _{t+j}-c_{t+j}.  \label{7}
\end{equation}

Once we have labour supply (\ref{Labour supply}) and $\Pi _{t+j}$ in terms
of output $Y_{t}$ and price mark-up $X_{t}$, we can rearrange (\ref{7}) to be%
\[
\Lambda _{t+j}=w_{t+j}^{\frac{\eta }{\eta -1}}c_{t+j}^{-\frac{1}{\eta -1}%
}+\left( 1-\frac{1}{X_{t+j}}\right) A_{t+j}k_{t+j}^{\alpha }w_{t+j}^{\frac{%
1-\alpha }{\eta -1}}c_{t+j}^{-\frac{1-\alpha }{\eta -1}}-c_{t+j}+T_{t+j}. 
\]

Linearize (\ref{6}), more details and parameters are from another tex file
`linearization of consumption'.

\begin{eqnarray*}
-\left[ C_{0}+\sum_{j=1}^{\infty }G_{cj}\right] \widehat{c}_{t}
&=&C_{w}E_{t-1}\widehat{w}_{t}+\frac{\theta }{\left( 1-\theta \beta \right)
\left( 1-\theta \right) }C_{X}\left[ \beta E_{t-1}\widehat{\pi }_{t+1}-%
\widehat{\pi }_{t}\right] +C_{A}\widehat{A}_{t}+C_{k}\widehat{k}_{t}+TE_{t-1}%
\widehat{T}_{t}... \\
&&+Rd\left[ E_{t-1}\widehat{R}_{t}+\widehat{d}_{t-1}-\widehat{\pi }_{t}%
\right] +\Lambda \sum_{j=1}^{\infty }\left[ \sum_{i=j}^{\infty }\left( \frac{%
1}{R}\right) ^{i}\widehat{\pi }_{t+j}\right] -\Lambda \sum_{j=1}^{\infty }%
\left[ \sum_{i=j}^{\infty }\left( \frac{1}{R}\right) ^{i}\widehat{R}_{t+j}%
\right] ... \\
&&+\sum_{j=1}^{\infty }G_{cj}\dsum\limits_{i=1}^{j}E_{t-1}\widehat{R}%
_{t+i}-\sum_{j=1}^{\infty }G_{cj}\dsum\limits_{i=1}^{j}E_{t-1}\widehat{\pi }%
_{t+i}+\sum_{j=1}^{\infty }G_{wj}E_{t-1}\widehat{w}_{t+j}+\frac{\theta }{%
\left( 1-\theta \beta \right) \left( 1-\theta \right) }\sum_{j=1}^{\infty
}G_{Xj}\left[ \beta E_{t-1}\widehat{\pi }_{t+1}-\widehat{\pi }_{t}\right] ...
\\
&&+\sum_{j=1}^{\infty }G_{Aj}\widehat{A}_{t+j}+\sum_{j=1}^{\infty
}G_{kj}E_{t-1}\widehat{k}_{t+j}+\sum_{j=1}^{\infty }G_{Tj}E_{t-1}\widehat{T}%
_{t+j}
\end{eqnarray*}

\subsection{Learning process}

For simplicity, we assume that retailers are fully rational so Phillips cure
is always valid. Also, we make central bank monetary policy unknown by the
households, so they make forecasts of interest rate by their PLM.

We assume at time $t-1$, we are at the steady state. We can get $k_{t}$ from
capital accumulation function and inflation $\pi _{t}$ from Philips curve.
Also, central bank set nominal interest rate to be $R_{t}$ according
toTaylor rule. At time $t$, the shock $A_{t}$ comes. From the consumption
function we derived above, households choose their consumption according to
state variables $k_{t}$, $A_{t}$, and their updated forecasts of $%
R_{t+i},w_{t+i}$ and $\pi _{t+i}$, $T_{t+i}$. Once households pin down the
consumption level $c_{t}$, \ labour supply is choosen according to the
labour supply function. After rearrangements of system of equations, we can
express $L_{t}$ as a function of $c_{t}$, and state variables $k_{t}$, $A_{t}
$. (we can express it in linearized form)%
\begin{eqnarray*}
c_{t}L_{t}^{\eta -1} &=&\frac{\left( 1-\alpha \right) k_{t}}{\alpha }%
L_{t}^{2\alpha -1}\left[ \frac{R_{t}}{\pi _{t}}-1+\delta \right] . \\
L_{t} &=&\left[ \frac{\left( 1-\alpha \right) k_{t}}{\alpha c_{t}}\left[ 
\frac{R_{t}}{\pi _{t}}-1+\delta \right] \right] ^{\frac{1}{\eta -2\alpha }}
\end{eqnarray*}

As households choose labour supply $L_{t}$, wage $w_{t}$ is derived and
output $Y_{t}$ is pinned down. We then know the capital in next period $%
k_{t+1}$ and this learning procedure goes on.

Now lets linearize the above equation%
\[
(\eta -\alpha )L^{\eta -\alpha }\widehat{L}_{t}=\left[ \frac{1-\alpha }{%
\alpha c}\right] \left( \frac{1}{\beta }\widehat{R}_{t}-\widehat{\pi }%
_{t}-\left( \frac{1}{\beta }-1-\delta \right) \widehat{c}_{t}\right) 
\]%
now we can recall Taylor rule%
\[
\widehat{R}_{t}=\alpha _{\pi }\widehat{\pi }_{t}+\alpha _{Y}\widehat{Y}_{t} 
\]%
from production function we have that 
\[
\widehat{Y}_{t}=\widehat{A}_{t}+\alpha \widehat{k}_{t}+\left( 1-\alpha
\right) \widehat{L}_{t} 
\]%
plug it into Taylor rule%
\[
\widehat{R}_{t}=\alpha _{\pi }\widehat{\pi }_{t}+\alpha _{Y}\left( \widehat{A%
}_{t}+\alpha \widehat{k}_{t}+\left( 1-\alpha \right) \widehat{L}_{t}\right) 
\]%
finally, plug Taylor rule in the equation for $\widehat{L}_{t}$%
\[
(\eta -\alpha )L^{\eta -\alpha }\widehat{L}_{t}=\left[ \frac{1-\alpha }{%
\alpha c}\right] \left( \frac{1}{\beta }\left( \alpha _{\pi }\widehat{\pi }%
_{t}+\alpha _{Y}\left( \widehat{A}_{t}+\alpha \widehat{k}_{t}+\left(
1-\alpha \right) \widehat{L}_{t}\right) \right) -\widehat{\pi }_{t}-\left( 
\frac{1}{\beta }-1-\delta \right) \widehat{c}_{t}\right) 
\]%
if we simplify the expression above we should get this guy%
\begin{equation}
\widehat{L}_{t}=\frac{1}{(\eta -\alpha )L^{\eta -\alpha }-\left[ \frac{%
1-\alpha }{\alpha c}\right] \frac{1}{\beta }\alpha _{Y}\left( 1-\alpha
\right) }\left[ \frac{1-\alpha }{\alpha c}\right] \left( \frac{1}{\beta }%
\left( \alpha _{\pi }\widehat{\pi }_{t}+\alpha _{Y}\left( \widehat{A}%
_{t}+\alpha \widehat{k}_{t}\right) \right) -\widehat{\pi }_{t}-\left( \frac{1%
}{\beta }-1-\delta \right) \widehat{c}_{t}\right)
\end{equation}%
Now we can use the above expression to solve for $\widehat{L}_{t}$ since it
is function only of the variables we know already, that is $\widehat{\pi }%
_{t}$, $\widehat{c}_{t}$, $\widehat{A}_{t}$ and $\widehat{k}_{t}.$

\end{document}
